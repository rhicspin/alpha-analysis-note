\documentclass[a4paper]{article}

\usepackage{amsmath}
\usepackage{fontspec}
\usepackage{polyglossia}
\usepackage[margin=20mm]{geometry}
\usepackage[colorlinks=true]{hyperref}
\urlstyle{same}
\usepackage{lineno}
\renewcommand\linenumberfont{\normalfont\tiny}
\linenumbers
\usepackage{wrapfig}
\usepackage{authblk}

\title{The Study of the Silicon Detector Response for p-Carbon Polarization Measurements at RHIC}

\author[1]{D.~Smirnov\thanks{d.s@plexoos.com}}
\author[2]{D.~Kalinkin}
\author[]{\ldots}

\affil[1]{Brookhaven National Lab}	
\affil[2]{ITEP}

\begin{document}

\maketitle

\abstract{At RHIC the measurements of proton beam polarization are carried out
by inserting a carbon target in the beam, and regestering the scattered carbon
ions with silicon detectors. In this note we present the results of the
analysis of the silicon detectors response to carbon atoms and alpha particles.}

\section{Motivation and Measurement Principles}

During the 2013 run we observed significant changes in the gain in some of the
silicon detector. The detector gain was monitored by taking calibration runs
when there was no beam in the machine. We use two radioactive source emmiting
alpha particles at known energies.


\section{Results}

in Figure~\ref{fig:}

\subsection{Dead layer size calculation}

\noindent Gain equality ($\mu$ - ADC, $E$ - measured energy):
\begin{equation}
\frac{\mu_{Am}}{E_{Am} - E_{DLAm}} = \frac{\mu_{Gd}}{E_{Gd} - E_{DLGd}}
\end{equation}
and constant stopping power approximation:
\begin{equation}
E_{DLAm} \simeq x_{DL} \lambda_{Am}\qquad
E_{DLGd} \simeq x_{DL} \lambda_{Gd}\qquad
\end{equation}
yield:
\begin{equation}
x_{DL} = \frac{\mu_{Gd} E_{Am} - \mu_{Am} E_{Gd}}{\mu_{Gd}\lambda_{Am} - \mu_{Am}\lambda_{Gd}}
\end{equation}
In the case of $\lambda_{Am}=\lambda_{Gd}$ this degenerates into a simple line fit

\section{Conclusions}

\end{document}
